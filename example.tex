\documentclass{cacthesis}

\begin{document}

	\frontmatter
	
	%%%%%%%%%%%%%
	%% Title page
	%%%%%%%%%%%%%
	\title{Fair Exchange Protocol over Bitcoin}
	\author{Leandro Rometsch}
	\date{October 19, 2020}
	\subject{Bachelor Thesis}
	
	\publishers{
		\small
		\begin{tabular}{r l}
			Supervisors: Benjamin Schlosser, \\ Prof. Sebastian Faust, Ph.D.
		\end{tabular}
	}
	\maketitle
	
	\tableofcontents
	
	\mainmatter
	
	\chapter{Introduction}
		\section{Section 1.1}
		\subsection{Subsection 1.1.1}
		\section{On Fairness}
		Exchange protocols aside, the obstacle when using the word "fairness" is that various interpretations among different individuals or even cultures exist. Fairness itself is something very relative and requires a precise definition when used in a scientific context. [REF TO "On Fairness in Exchange Protocols"] stated that it is crucial for defining fairness in a specific context to focus on what fairness is and not on how to obtain it. \\
        Shifting the construct of fairness to Exchange Protocols, it is easy to state that a "Fair" Exchange Protocol requires to ensure that no party can gain a significant advantage over the other party. Different explanations of what is meant with a "significant advantage" result in different fairness notions. \\
        The fundamental definition commonly used for today's state of the art Fair Exchange Protocols (like [REF, REF, ...]) is the following: At the end of every exchange protocol run, either all involved parties obtain their expected information, or none of them receives anything. \\
        This definition might be satisfying in some scenarios, but "advantage" is defined minimally and only focuses on the actual exchange result here. In other words, this definition could be interpreted as some superficial notion of fairness. Let us assume we have the following scenario: B, the buyer, and M, the merchant, use an (in the sense of the above-defined superficial notion of fairness) Fair Exchange Protocol to trade some coins for a digital item. To allow B to verify that the digital item is correct, B needs to run a computationally high demanding reviewing procedure. If the checking procedure fails, proof of misbehavior regarding M is generated, and B can get a full refund. If the checking procedure approves, B cannot get his money back. \\
        Depending on the interpretation of fairness, one could argue that M has an advantage over B because B is forced to invest significantly more computational time, even when B cannot receive the digital item afterward. Evaluating the fairness of a protocol is, therefore, only possible under certain assumptions. \\
        While explaining "advantage" in the context of Fair Exchange Protocols [REF TO "On Fairness in Exchange Protocols"] comes up with three main aspects that make up different notions of fairness. (1) Fairness, directly relating to the items being exchanged during the protocol. (2) Timeliness, relating to determining the progress of a protocol and the option to abort it at some point while preserving fairness. (3) Abuse-freeness, in that sense, that no single party alone can prove to an outside one, that he has the power to terminate or complete the protocol successfully. A specific notion of fairness might consider one aspect stronger, weaker, or even irrelevant. \\
        Another common notion of fairness is "Strong Fairness", sometimes referred to as perfect fairness. The conventional definition for "Strong Fairness" is that a party cannot leave the protocol with even a small advantage over the opponent. Again, this definition is not absolute and only valuable if "advantage" is adequately defined, for example, by the three aspects mentioned. Additionally, in protocols providing "Strong Fairness" it should always be possible for the buyer to retrieve the original expected item in case of a problem, without needing the misbehaving party to cooperate. Usually, this is archived by heavyweight Trusted Third Parties. However, there exist optimistic protocols using transparent Trusted Third Parties [REF, REF, REF] that archived to maintain their notion of "Strong Fairness". \\
        "Weak Fairness" is given if an honest party can prove his honesty to an external judge, like a Trusted Third Party. If the potentially misbehaving party is not cooperating, the honest party receives some replacement. \\
        Finally, there are Fair Exchange Protocols that merge their notion of fairness with probability. These are usually protocols like [Data Trading Based on Bitcoin REF] that depend on techniques that offer a probability of fairness (e.g., cut-and-choose) based on the chosen security parameters. This approach is challenging to classify in our existing idea of fairness. The most logical one is that we recognize "Probabilistic Fairness" as a relativization to already defined notions of fairness since most definitions can be easily shifted into a probabilistic context. Although it is a straight forward answer, this raises the problem if notions like "Perfect Fairness" can be seen in a probabilistic scenario, resulting in some form of "Probabilistic Perfect Fairness". Although this is a very relevant topic, we will leave this question open to be answered by other publications.


		
	\chapter{Chapter 2}
	
	%\bibliographystyle{}
	%\bibliography{}
	
	\appendix
\end{document}