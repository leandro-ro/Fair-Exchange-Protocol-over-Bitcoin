\documentclass{cacthesis}

\begin{document}

	\frontmatter
	
	%%%%%%%%%%%%%
	%% Title page
	%%%%%%%%%%%%%
	\title{Fair Exchange Protocol over Bitcoin}
	\author{Leandro Rometsch}
	\date{October 19, 2020}
	\subject{Bachelor Thesis}
	
	\publishers{
		\small
		\begin{tabular}{r l}
			Supervisors: Benjamin Schlosser, \\ Prof. Sebastian Faust, Ph.D.
		\end{tabular}
	}
	\maketitle
	
	\tableofcontents
	
	\mainmatter
	
	\chapter{Introduction}
        Sum up the basics of an data exchange protocol. Talk about the possibilities of smart contracts, how they can realize data exchange protocols and why Bitcoin based smart contracts might be an good alternative to Ethereum based ones. 
        
        \section{Current solutions}
        Briefly write about the interesting fair exchange protocols I reviewed in the research phase. What is the most interesting aspect about them? What can we learn from them?
        
        \section{Contribution}
        Understanding  the current concept of fairness (and the problems relating to it). \\
        What is necessary to convert a script like FairSwap onto a potentially cheaper Bitcoin variant and why it is challenging.
        FairSwap inspired fair exchange protocol that runs cheaper on Bitcoin Script.

        \section{Related Work}
        How Bitcoin evolved to Bitcoin Cash. What are the key differences between Bitcoin, Bitcoin Cash and Ethereum in terms of smart contracts.
    
    \chapter{Preliminaries}
        \section{Bitcoin Transactions}
        A broad overview on how Bitcoin Transactions work an how they are structured.
        
        \section{Bitcoin Script}
        Explain Bitcoin Script basics here. What important opcodes are there? Give tx examples.
        
        \section{FairSwap}
        Explain FairSwap in more detail. What level of fairness is archived? Write about the protocols cryptography building blocks (Merkle Proofs, concise proof of misbehavior, ...). Sum up how the protocol got implemented and give a statement about its performance. Do not mention any approaches on how to move this protocol to Bitcoin yet.
        
        \section{Building Blocks}
        \subsection{Merkle Tree}
        \subsection{Lockup Tx}
        \subsection{Claim-or-refund Tx}
        \subsection{Refund Tx}

    \chapter{Fairness in Exchange Protocols}
        Bentov and Kumaresan state that a fair exchange protocol is a particular subcase of secure multiparty computation \cite{10.1007/978-3-662-44381-1_24}. Numerous multiparty computation protocols want to achieve a notion of security that is requiring a form of fairness. One common way of defining fairness, in this case, is \textit{if one party receives their expected output, then so do all} \cite{10.1007/978-3-540-79263-5_8}. Cleve showed in \cite{10.1145/12130.12168} that it is impossible to achieve this notion of fairness when there is no honest majority among the protocol participants. Let us carry this knowledge into the context of a two-party exchange protocol: At the point where there is one dishonest party involved, it is impossible to achieve the mentioned form of fairness. Consequently, it is crucial to specify acceptable alternative notions of fairness that are specially tailored for the two-party case. 
        
        \section{Challenges and the Superficial Notion of Fairness}
		The obstacle when using the construct of \textit{fairness} is that various interpretations among different points of view exist. Markowitch et al. stated that it is crucial for defining fairness in exchange protocols to focus on what fairness is and not on how to obtain it \cite{10.1007/3-540-36552-4_31}. It is obvious to claim that a \textit{fair} exchange protocol requires that no party gains a \textit{significant advantage} over the other party under the limitation of particular assumptions. Different explanations of what is meant with a \textit{significant advantage} result in different fairness notions. \\
        The primary definition commonly used among publications around fair exchange protocols (like \cite{10.1145/266420.266426} \cite{asyncOptiFairEx1998} \cite{remarksOnFairEx2000})  is the following: \textit{At the end of every exchange protocol run, either all involved parties obtain their expected information/asset, or none of them receives anything.} This definition might be satisfying in some scenarios, but \textit{advantage} is defined minimally and only focuses on the actual exchange result here. In other words, this definition could be interpreted as some superficial notion of fairness.
        
        \section{Imaginary Scenario}
        Let us imagine the following scenario: $B$, the buyer, and $S$, the seller, use a fair exchange protocol $P$ to trade some digital coins for a digital item. The high-level procedure of $P$ is as follows: $B$ locks up digital coins that $B$ is willing to spend for a particular digital item $D$ that satisfies the predicate $\sigma$ s.t. $\sigma\left( D\right) =1$. The digital coins can be redeemed by $S$ if $S$ can deliver a digital item $D'$ s.t. $\sigma\left( D'\right) =1$. If $S$ can do this, $B$ can be sure that $D' = D$ and both parties are satisfied. If $S$ is not responding with a matching $D'$, the previously locked-up coins will be available for $B$ again after a pre-defined timeframe $t$, and $S$ cannot claim them anymore. \\
        In the sense of the above-defined superficial notion of fairness, this protocol is fair because there are only two possible outcomes of $P$. Either the exchange succeeded or non of the parties receives anything. 
        Although $P$ is certainly fair under this definition, $B$ got a significant disadvantage. $B$ cannot use the coins for something else during the execution of $P$, and a malicious $S$ could use this knowledge, e.g., for a DOS attack. One could argue that this is not fair because one party got a \textit{significant advantage} over the other party. Therefore giving definitions for different notions of fairness comes hand-in-hand with defining advantage. This is, unfortunately, missing in our mentioned superficial notion of fairness. 
        
        \section{Defining "Advantage"}
        While explaining \textit{advantage} in the context of fair exchange protocols Markowitch et al. come up with three main aspects that make up different notions of fairness \cite{10.1007/3-540-36552-4_31}. (1) \textbf{Fairness}, directly relating to the items being exchanged during the protocol. (2) \textbf{Timeliness}, relating to determining the progress of a protocol and the option to abort it at some point while preserving fairness. (3) \textbf{Abuse-freeness}, in that sense, that no single party alone can prove to an outside one that he has the power to terminate or complete the protocol successfully. A specific notion of fairness might consider one aspect stronger, weaker, or even irrelevant. \\
        Projecting these aspects on our imaginary exchange protocol $P$: (1) is undoubtedly fulfilled because either the exchange succeeded or non of the parties receives anything. (2) is only partially fulfilled because $S$ might be able to abort by not responding, but $B$ must wait until the coins are unlocked. $B$ is also not able to determine the progress of the protocol. $S$ might be preparing the transmission of $D'$ or already aborted. Either way, the protocol is time-wise limited by the pre-defined timeframe $t$. (3) is not fulfilled. If $S$ indeed owns $D'$ s.t. $\sigma\left( D'\right) =1$, $S$ can prove to an outside party that $S$ has the power to terminate or complete $P$ successfully, at least inside the timeframe until $B$ can unlock the coins again. $S$ can abuse $P$ for the only purpose of blocking $B$'s coins for a particular time - even if $S$ does not know $D$.
        
        \section{Strong Fairness, Weak Fairness}
        Another common notion of fairness is \textit{Strong Fairness}, sometimes referred to as perfect fairness. Pagnia and Gaertner created an early formal definition for \textit{Strong Fairness} in exchange protocols \cite{Pagnia99onthe}. They are essentially stating that both parties got some expectation from the goods being exchanged, much like the predicate $\sigma$ of our imaginary protocol $P$. \textit{At the end of the protocol run, either both parties expectations are being fulfilled after the exchange protocol, or no information about the other party's respective good is gained.} This sounds a lot like the previously mentioned superficial notion of fairness. Indeed it is equivalent, but the modern usage of \textit{Strong Fairness} goes beyond this: \textit{A single party cannot leave the protocol with even a small advantage over the opponent} \cite{DELGADOSEGURA2020832}. Again, this is not absolute and only valuable if \textit{advantage} is adequately defined, for example, by the three aspects mentioned. \\
        It is worth mentioning that some publications expect from their notion of \textit{Strong Fairness} that it should always be possible for the buyer to retrieve the original expected item in case of a problem, without needing the misbehaving party to cooperate \cite{DELGADOSEGURA2020832}. For our imaginary protocol $P$, this is not the case and would require a different approach. Usually, this is achieved by heavyweight Trusted Third Parties ($TTP$). This $TTP$ could take $D$ from $S$ as an initial step in $P'$ and check if the expectation $\sigma$ is fulfilled. B only proceeds if the $TTP$ approves. The protocol $P'$ could then run as defined in $P$ with the small addition that if $S$ is not providing $D$ until $t$ by itself, the $TTP$ will send a copy of $D$ to $B$ and finish the exchange. However, there exist optimistic protocols using transparent Trusted Third Parties that are used to maintain this notion of \textit{Strong Fairness} \cite{6982058} \cite{10.1007/3-540-36552-4_31}. \\
        While in \textit{Strong Fairness} usually, an honest party does not need to get active to prove its honesty in case of a dispute, in \textit{Weak Fairness} an honest party can prove his honesty to an external judge, like a TTP \cite{10.1007/3-540-36552-4_31}. The external judge can then evaluate the dispute. If the potentially misbehaving party is not cooperating, the honest party receives some compensation. A protocol achieving \textit{Weak Fairness} is therefore dependent on the external judge's honesty.
        
        \section{Probabilistic Fairness}
        Finally, there are Fair Exchange Protocols that merge their notion of fairness with probability. These are usually protocols that depend on techniques that offer a probability of fairness (e.g., cut-and-choose in \cite{DELGADOSEGURA2020832}) based on the chosen security parameters. This approach is challenging to classify in our existing idea of fairness. The most consistent one is that we recognize \textit{Probabilistic Fairness} as a relativization to already defined notions of fairness since most definitions can be easily shifted into a probabilistic context. Let us do this for our previously mentioned superficial notion of fairness: \textit{For an adequate security parameter k, at the end of the exchange protocol run there must be an overwhelming probability that either all parties obtain their expected information or non of the parties obtain anything.} \\
        To further elaborate on this, we project this again on our imaginary protocol $P$. We construct a new protocol $P''$ that replaces the previous predicate $\sigma$ with a cut-and-choose procedure. $S$ now needs to provide a $D'$ that is split up into $n$ parts. $B$ chooses $k < n$ indices $K$. These indices are not known by $S$. For a successful exchange, $S$ needs to provide a $D'$ s.t. $\forall i\in K:\sigma'\left( D_{i}'\right) =1$. \\

        
        %\section{Fairness techniques} 
        % Gradiual Release, Claim-or-refund, cut-and-choose.
        % maybe https://eprint.iacr.org/2005/370.pdf also interesting
		
	\chapter{Our construction}
	   
	   \section{FairSwap to Bitcoin}
        What is necessary to move FairSwaps Ethereum implementation to Bitcoin Script and why it is not straight forward? How did we solve the problems on a high level? 
        
        \section{The protocol}
        Combine the building blocks mentioned in the Preliminaries to form a formal version of the protocol.
        
        \section{Security and Fairness}
	    Informal security analysis - discuss various attack vectors. Connect the findings of the Fairness section with the actual fairness of the finish protocol.
	    
	    \section{Implementation and Performance}
	    
	    
	\chapter{Conclusion}
    
	
	\newpage
    \bibliographystyle{unsrt}
    \bibliography{ref}
	
	\appendix
\end{document}